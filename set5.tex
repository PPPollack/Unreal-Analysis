\chapter*{$p$-Set \#6}
\addcontentsline{toc}{chapter}{Set \#6}
\markboth{Set \#6}{Set \#6}


\section*{Enter Cauchy}
Let $K$ be a field equipped with an absolute value $|\cdot|$. If $\{x_n\}$ converges to $x$ in $(K,|\cdot|)$, then for any real number $\epsilon > 0$, all terms far enough out in the sequence $\{x_n\}$ are within $\frac{1}{2}\epsilon$ of $x$. By the triangle inequality, all such terms are within $\epsilon$ of each other. That is:
\vskip 0.05in
\fbox{
    \parbox{4.2in}{
        For each $\epsilon > 0$, there is a positive integer $N$ with \[ |x_n - x_m| < \epsilon \quad\text{whenever $n, m \ge N$.}\]
    } (C)
}
\vskip 0.05in
Any sequence $\{x_n\}$ with property (C) is called a \textsf{Cauchy sequence}.\index{Cauchy sequence}

\begin{prob}\label{prob:68} If $|\cdot|$ is non-Archimedean, then $\{x_n\}$ is Cauchy $\Longleftrightarrow$ $|x_{n+1}-x_n| \to 0$. 

{\scriptsize This need not hold if $|\cdot|$ is Archimedean; a counterexample is provided by the partial sums of the harmonic series in $(\Q,|\cdot|_{\infty})$.}
\end{prob}

\begin{prob}\label{prob:70} The sequence $\{2^{5^n}\}$ is Cauchy in $(\Q, |\cdot|_5)$. 
%It does not converge in $(\Q,|\cdot|_5)$ but does converge in $(\Q_5,|\cdot|_5)$, to a $4$th root of $1$.
\end{prob}


\begin{prob}[a common calculus Cauchy claim]\label{prob:68point5} Let $(K,|\cdot|)$ be a valued field. If $\{x_n\}$ is a Cauchy sequence in $K$, then $\{|x_n|\}$ is a bounded sequence of real numbers.
\end{prob}

\begin{prob}[and another] \label{prob:69} Let $(K,|\cdot|)$ be a valued field. If a sequence $\{x_n\}$ of elements of $K$ is Cauchy, and some subsequence of $\{x_n\}$ converges to $x \in K$, then $\{x_n\}$ converges to $x$.
\end{prob}

\vspace{-0.22in}
\testrule 
If $\R$ is equipped with the usual absolute value, then every Cauchy sequence in $\R$ converges to an element of $\R$. This need not be the case for a general valued field. For example, the sequence of rational numbers
\[ x_1 =1, \quad x_2 = 1.4, \quad x_3 = 1.41, \quad x_4 = 1.414,\quad \dots, \]
obtained by successively truncating the decimal expansion of $\sqrt{2}$, is Cauchy in $(\Q,|\cdot|_{\infty})$ but does not converge to any element of $\Q$, since $\sqrt{2}\notin \Q$. Coping with this unsettling scenario was one of the motivations behind the invention (discovery?) of the real numbers in the first place!

Disconcerting examples of this same kind can also be found when $K=\Q$ and $|\cdot| = |\cdot|_p$. Take any sequence $\{c_k\}_{k\ge 0}$ from $\{0,1,\dots,p-1\}$ that is not eventually periodic. Then $x_n = \sum_{k=0}^{n} c_k p^k$ defines a Cauchy sequence in $(\Q,|\cdot|_p)$. To check the Cauchy condition (C), we may assume that $m < n$ (why?). Then  
\[ |x_n - x_m|_p = |c_{m+1} p^{m+1} + c_{m+2} p^{m+2} + \dots + c_n p^{n}|_p < p^{-m},\]
which is smaller than $\epsilon$ once $m > \frac{\log(1/\epsilon)}{\log{p}}$. So (C) is satisfied for any $N > \frac{\log(1/\epsilon)}{\log{p}}$. But $\{x_n\}$ cannot converge to an element of $\Q$, on account of Exercise \ref{ex:rational}.

To get out of this mess, we need to fill in the following blank: $\R$ is to $(\Q,|\cdot|_{\infty})$ as $\underline{\hphantom{nonsense}}$ is to $(\Q,|\cdot|_{p})$. The answer here turns out to be $\Q_p$! But \dots what is $\Q_p$?
\testruletwo

\section*{The $p$-adic Numbers, At Last!}

We define the $\textsf{field of $p$-adic numbers}$\index{p-adic@$p$-adic number|see{$\Q_p$, field of $p$-adic numbers}}, denoted $\Q_p$\index{$\Q_p$, field of $p$-adic numbers!introduction and basic properties},  as the fraction field of  $\Z_p$. Since $\Z_p$ has characteristic $0$, so does $\Q_p$, and thus $\Q\subset \Q_p$.
% As fields of characteristic $0$, each $\Q_p$ contains a canonical copy of $\Q$.

\begin{prob}\label{prob:63} $\Q_p = \bigcup_{n\ge 0} p^{-n} \Z_p =  \Z_p[1/p]$.
\end{prob}

\begin{prob}\label{prob:64} Every nonzero $x \in \Q_p$ admits a unique expression in the form $p^v u$, where $v$ is an integer and $u$ is a unit in $\Z_p$. \end{prob}

For $x \in \Q_p^{\times}$, we set $v_p(x)=v$, where $v$ is the integer from Problem \ref{prob:64}. In order to have $v_p$ defined on all of $\Q_p$, we let $v_p(0)=\infty$. (Compare with the definition of $v_p$ on Set \#1.)

\begin{prob}\label{prob:65} When $x \in \Q$, we have defined $v_p(x)$ twice: once on Set \#1 and again just now, since $x$ is also an element of $\Q_p$. 

Check that when $x \in \Q$ our two definitions of $v_p(x)$ agree (so our sin is venial, rather than mortal). Furthermore, if we set $|x|_p = p^{-v_p(x)}$ for $x \in \Q_p$, then $|\cdot|_p$ defines a non-Archimedean absolute value on $\Q_p$. (We continue to call $v_p$ the \textsf{$p$-adic valuation}\index{p-adic@$p$-adic!valuation} and $|\cdot|_p$ the \textsf{$p$-adic absolute value}.)\index{absolute value!$p$-adic}\index{p-adic@$p$-adic!absolute value}
\end{prob}

\begin{prob}\label{prob:66} $\Z_p = \{x \in \Q_p: |x|_p \le 1\}$. (That is, $\Z_p=\Dd_{\le 1}(0)$ in $\Q_p$.) \\
Also, $\Z_p^{\times} = \{x \in \Q_p: |x|_p = 1\}$.
\end{prob}
\begin{prob}[$\Z_p$ is compact]\index{$\Z_p$, ring of $p$-adic integers!is compact}\label{prob:67}\label{prob:zpcompact} Let $x_1, x_2, x_3, \dots$ be a sequence of elements of $\Z_p$. Infinitely many $x_n$ share the same mod $p$ component, say $a_1\bmod{p}$. Among these, infinitely many share the same mod $p^2$ component, $a_2\bmod{p^2}$. Etc. Thus:      $x_1,x_2,x_3,\dots$ contains a subsequence converging to $(a_1\bmod{p}, a_2\bmod{p^2}, a_3\bmod{p^3}, \dots) \in \Z_p$. 
\end{prob}




\curious

\begin{prob}[Stern]\label{prob:71} Recall that $\log\frac{1}{1-t} = t+\frac{1}{2}t^2 +\frac{1}{3}t^3+  \dots$ whenever $t$ is a complex number with $|t| < 1$ (usual absolute value). Exponentiating, $\e^{T} \e^{T^2/2} \e^{T^3/3} \cdots =\frac{1}{1-T}$, as formal power series. Expanding and comparing coefficients of $T^p$ shows that $|1 - \frac{1}{p!} - \frac{1}{p}|_{p} \le 1$. Hence, $(p-1)!\equiv -1\pmod{p}$ (\textsf{Wilson's theorem}).\index{Wilson's theorem}
\end{prob}


\begin{prob}\label{prob:72} For each prime $p$ and each $a\in \Z$ coprime to $p$, put $q_p(a) = \frac{a^{p-1}-1}{p}$. By Fermat's little theorem, $q_p(a) \in \Z$. 

Prove: If $a$ and $b$ are both coprime to $p$, then $q_p(ab) \equiv q_p(a) + q_p(b) \pmod{p}$.
\end{prob}








