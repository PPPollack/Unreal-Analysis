
\chapter*{Solutions to Set \#4}
\addcontentsline{toc}{chapter}{Solutions to Set \#4}
\markboth{Solutions to Set \#4}{Solutions to Set \#4}
\label{set3sols}


\begin{sol}{prob:41} We have $1+\e^{T} + \e^{2T} + \dots + \e^{(n-1)T} = \sum_{0 \le j < n} \sum_{k\ge 0} \frac{(jT)^{k}}{k!} 
=\sum_{k\ge 0} \left(\sum_{0 \le j < n} j^k\right) \frac{T^k}{k!} = \sum_{k\ge 0} S_k(n) \frac{T^k}{k!}$.
\end{sol}

\begin{sol}{prob:42} Summing the finite geometric series,
\[ \sum_{0 \le j < n} \e^{jT} = \sum_{0 \le j < n} (\e^{T})^{j} = \frac{\e^{nT}-1}{\e^{T}-1} = \frac{\e^{nT}-1}{T} \cdot \frac{T}{\e^{T}-1}.\]
Therefore, by Problem \ref{prob:41},
\begin{align*} \sum_{k \ge 0} S_k(n) \frac{T^k}{k!} &= \frac{\e^{nT}-1}{T} \cdot \frac{T}{\e^{T}-1}\\
&= \left(\sum_{r \ge 0} \frac{n^{r+1} T^{r}}{(r+1)!}\right) \left(\sum_{s\ge 0} B_s \frac{T^s}{s!}\right) \\
&= \sum_{k\ge 0} \frac{T^k}{k!} \sum_{\substack{r+s=k \\ r,s\ge 0}} B_s \frac{(r+s)!}{(r+1)! s!} n^{r+1}.
\end{align*}
Comparing coefficients of $\frac{T^k}{k!}$ gives
\begin{align*} S_k(n) &= \sum_{\substack{r+s=k\\ r,s\ge 0}} B_s \frac{(r+s)!}{(r+1)! s!} n^{r+1} = \sum_{0 \le r \le k} \frac{B_{k-r}}{r+1} \binom{k}{r} n^{r+1}.\index{Bernoulli numbers!Faulhaber's formula}\index{Faulhaber's formula}\end{align*}

\begin{rmk} If you stare carefully, you will notice this argument takes for granted the identities $\e^{jT} = (\e^{T})^j$ (for $j=0,1,2,3\dots$). Here what is important is not that these identities hold for real numbers $T$ (which is familiar from calculus), but that they hold as identities of formal power series in the indeterminate $T$. 

To effect a proof, write \[ (\e^{T})^j = \sum_{k_1,\dots,k_j\ge 0} \frac{1}{k_1! \cdots k_j!} T^{k_1+\dots+k_j} = \sum_{k \ge 0} \frac{T^k}{k!} \sum_{k_1+\dots+k_j=k} \binom{k}{k_1,k_2,\dots,k_j}.\] By the multinomial theorem, $\sum_{k_1+\dots+k_j=k} \binom{k}{k_1,k_2,\dots,k_j}= j^k$. Therefore, $(\e^{T})^j = \sum_{k \ge 0} \frac{(jT)^k}{k!} = \e^{jT}$. 

% Even though invoking the ``real'' world was not necessary in this instance, the technique of transmogrifying an identity of real (or complex) numbers into an identity of formal power series is frequently useful.

We could also have proved the required identities by leveraging our prior knowledge of the ``real world.''
% Fortunately, the (obvious) real identities can be shown to imply the (not-so-obvious) formal identities! 
Expand (formally) $\e^{jT} - (\e^{T})^j = \sum_{k \ge 0} c_k T^k$ for some coefficients $c_k$. The exact same transformations you use to put the left side into the form of the right will show that
\[ (\e^{x})^j - \e^{jx} = \sum_{k\ge 0} c_k x^k \quad\text{for all \underline{real} numbers $x$}. \]
Here all manipulations with real numbers can be justified by citing absolute convergence of the relevant series. (Operations on formal power series are \emph{defined} to mirror operations that can be performed on numerical series in the presence of sufficiently good convergence.) Since $(\e^{x})^j - \e^{jx}=0$ for all real numbers $x$, the series $\sum_{k\ge 0} c_k x^k$ converges everywhere to $0$. This forces each $c_k=0$, which in turn shows that $\e^{jT} = (\e^{T})^j$ formally. While gratuitous in this instance, the principle that an identity of real numbers can (often) be transmogrified into an identity of formal power series is frequently useful.
\end{rmk}
\end{sol}

\begin{sol}{prob:43}\index{Bernoulli numbers!vanish for odd indices $>1$} All of the claimed equalities are straightforward to verify, including the invariance of $T\coth{T}$ under the substitution $T\mapsto -T$ (provided we accept that $\e^{2T} \e^{-2T}=1$, which can be proved by the method of the preceding Remark). Writing down the power series for $-T\coth(-T)$ and $T\coth(T)$, we conclude that
\[ -T + \sum_{k\ge 0} B_k \frac{(-2T)^{k}}{k!} = T + \sum_{k\ge 0} B_k \frac{(2T)^{k}}{k!}. \]
When $k$ is odd and larger than $1$, comparing coefficients of $T^k$ on both sides shows that $B_k=0$. When $k=1$, the same reasoning gives $-1 -2B_1 = 1 + 2B_1$, leading to $B_1 = -\frac{1}{2}$ (as asserted in our table).

\begin{sol}{prob:alternatingsigns}\index{Bernoulli numbers!alternate in sign} Starting from $T\coth{T} = T + \sum_{k\ge 0} B_k \frac{(2T)^{k}}{k!}$, divide by $T$ and substitute $B_0=1$, $B_1=-\frac{1}{2}$ and $B_k=0$ for odd $k>1$. This provides the expansion claimed for $\coth{T}$. That $\frac{\mathrm{d}}{\mathrm{d}T}\coth{T} = 1-\coth^2{T}$ can be checked directly.

To ease notation, write \[ \coth{T} = \frac{1}{T} + \sum_{k\ge 1} c_k T^{2k-1}.\] As $c_k$ and $B_{2k}$ share the same sign, we will be done if we show that $(-1)^{k+1} c_k > 0$ for each $k \in \Z^{+}$. Assuming this claim fails, let $K$ be the minimal counterexample. For later use, note that $K > 1$ (since $c_1 = 2B_2 = \frac{1}{3} > 0$). Differentiating the last displayed equation,
\[ \frac{\mathrm{d}}{\mathrm{d}T}\coth{T} = -\frac{1}{T^2} + \sum_{k\ge 1} (2k-1) c_k T^{2k-2}. \]
The right-hand Laurent series has $T^{2K-2}$ appearing with coefficient $(2K-1) c_K$. On the other hand, the coefficient of $T^{2K-2}$ in $\frac{\mathrm{d}}{\mathrm{d}T}\coth{T}$ is the same as its coefficient in $-\coth^2{T}$, since $\frac{\mathrm{d}}{\mathrm{d}T}\coth{T} = 1 - \coth^2{T}$ and $K>1$. That coefficient is $-(2c_K+\sum_{i+j=K,~i,j\ge 1} c_i c_j)$.  Therefore,
$(2K+1) c_K = -\sum_{i+j=K,~i, j\ge 1} c_i c_j$, and
\[ (-1)^{K+1} c_K = \frac{(-1)^{K}}{2K+1} \sum_{\substack{i+j=K\\i, j\ge 1}} c_i c_j = \frac{1}{2K+1} \sum_{\substack{i+j=K\\i, j\ge 1}} (-1)^{i+1} c_i\cdot (-1)^{j+1} c_j > 0.\]
We use in the last step that $(-1)^{i+1}c_i > 0$ and $(-1)^{j+1} c_j > 0$, since $i$ and $j$ are positive integers smaller than $K$.
\end{sol}


\begin{rmk} A more satisfying explanation for why the even-indexed Bernoulli numbers alternate in sign is found in Euler's formula\index{Bernoulli numbers!in terms of $\zeta$-values}
    \[ B_{2k} = (-1)^{k+1} 2\zeta(2k) \cdot \frac{(2m)!}{(2\pi)^{2m}}, \]
    where $\zeta(s):=\sum_{n\ge 1} \frac{1}{n^s}$ is the \textsf{Euler--Riemann zeta function}. This remarkable relation pins down $B_{2k}$ rather precisely as a real number. Indeed, when conjoined to Stirling's estimate on factorials, it implies that
    \[ \lim_{k\to\infty} \frac{|B_{2k}|}{4\sqrt{\pi k} (\frac{k}{\pi \e})^{2k}} = 1.\]
Unfortunately, Euler's formula does not contain any obvious information about the \emph{number-theoretic} properties of $B_{2k}$.
    
    We will not prove Euler's result here. Interested readers are referred to the exquisitely written textbook of Ireland and Rosen for a characteristically elegant treatment \cite[pp.\ 231--232]{IR}.
\end{rmk}
\end{sol}



\begin{sol}{prob:33} Information on the first ten        partial sums is collected below.

{
\centering
\setlength{\tabcolsep}{16pt}
    \begin{tabular}{rr}
        $n$ & $\sum_{1\le k\le n}2^k/k$ \\\midrule
        1 & 2 \\
        2 & $2^2$  \\
        3 & $2^2 \cdot 5/3$ \\
        4 & $2^2 \cdot 8/3$\\
        5 & $2^8 \cdot 1/15$  \\
        6 & $2^5\cdot 13/15$ \\
        7 & $2^5 \cdot 151/105$ \\
        8 & $2^{13} \cdot 1/105$ \\
        9 & $2^9 \cdot 83/315$ \\
        10 & $2^{10} \cdot 73/315$
    \end{tabular}\par
}
    %\caption{Caption}

    
Already this limited data suggests that $v_2(\sum_{1\le k\le n} 2^k/k)$ tends to infinity with $n$ (equivalently, that $\sum_{k=1}^{\infty}2^k/k = 0$ in $(\Q,|\cdot|_2)$). Being a bit bolder, we might conjecture that $v_2(\sum_{1\le k\le n} 2^k/k)$ is bounded below by a function ever-so-slightly smaller than $n$. For the resolution of both conjectures, see the solution to Problem \ref{prob:padiclumber1}.
\end{sol}


\begin{sol}{prob:44}\index{$\Q_p$, field of $p$-adic numbers!element has eventually periodic base $p$ expansion
iff rational} We need the following lemma.

\begin{lem} For each $x \in \Z_{(p)}$, there is a $d \in \{0,1,2,\dots,p-1\}$ with $x-d \in p\Z_{(p)}$.
\end{lem}
\begin{proof} Write $x=a/b$ where $a, b \in \Z$ and $p\nmid b$. Choose $B\in \Z$ with $Bb\equiv 1\pmod{p}$, and select $d\in \{0,1,\dots,p-1\}$ with $d\equiv aB\pmod{p}$. Then $db \equiv aBb\equiv a\pmod{p}$, and $x-d = p \frac{(a-db)/p}{b} \in p\Z_{(p)}$. 
\end{proof}

With the lemma in hand, we can express $r$ in the desired form by the algorithm of Problems \ref{prob:31} and \ref{prob:32}. Specifically, let $x_0 = r$, and for $n=0,1,2,3,\dots$, select $d_n \in \{0,1,2,\dots,p-1\}$ so that $x_n = d_n + p x_{n+1}$ for some $x_{n+1} \in \Z_{(p)}$. Then $r = \sum_{k\ge 0} d_k p^k$. 

Having shown existence we turn to uniqueness. Suppose $r =  \sum_{k\ge 0} d_k p^k = \sum_{k\ge 0} d_k' p^k$ with all $d_k, d_k' \in \{0,1,\dots,p-1\}$. Assume $\{d_k\}$ and $\{d_k'\}$ do not coincide, and let $k_0$ be the smallest nonnegative integer with $d_{k_0} \ne d_{k_0}'$. Then $0 = r-r' = \sum_{k\ge k_0} (d_k - d_k') p^k$, and 
\[ (d_{k_0}' -d_{k_0}) p^{k_0}= \sum_{k > k_0} (d_k - d_k') p^k.\]
So if we set $s_n = \sum_{k_0 < k \le n} (d_k-d_k')p^k$, then $s_n \to (d_{k_0}' -d_{k_0}) p^{k_0}$ in $(\Q,|\cdot|_{p})$. That is,
\begin{equation}\tag{*} |s_n - (d_{k_0}' -d_{k_0}) p^{k_0}|_p \to 0.\end{equation} However, $|s_n|_p \le \max_{k_0 < k \le n} |(d_k-d_k')p^k|_p \le p^{-k_0-1}$, while $|(d_{k_0}' -d_{k_0}) p^{k_0}|_p = p^{-k_0}$. So by ``survival of the greatest,'' 
\[ |s_n - (d_{k_0}' -d_{k_0}) p^{k_0}|_p = p^{-k_0} \]
for every $n > k_0$, contradicting (*).

Finally we prove $\{d_k\}$ is eventually periodic. It suffices to show that in the algorithm of Problems \ref{prob:31} and \ref{prob:32}, there will always be  nonnegative integers $m < n$ with $x_{m} = x_{n}$. In that case $\{d_k\}$ repeats, from $k=m$ onwards, with (not necessarily minimal) period $n-m$.

Write $r=a/b$ where $p\nmid b$. We will base our proof on two claims:
\begin{enumerate}
\item[(a)] each $x_n \in \frac{1}{b}\Z$,
\item[(b)] each $|x_n|_{\infty}\le M$, where $M:= \max\{2,|r|_{\infty}\}$.
\end{enumerate}

Claim (a) is obvious when $n=0$, since $x_0 = r= a/b$. Suppose that $x_n\in \frac{1}{b}\Z$. Then $b p x_{n+1} = bx_n - b d_n \in \Z$. As $x_{n+1} \in \Z_{(p)}$, we also have $|bp x_{n+1}|_p = |bp|_p |x_{n+1}|_{p} \le p^{-1}$. So in fact $bp x_{n+1} \in p\Z$, and $x_{n+1} \in \frac{1}{bp} (p\Z) = \frac{1}{b}\Z$. This gives (a). Since $x_0=r$, (b) is trivial when $n=0$. If $|x_n|_{\infty} \le M$, then \[ |x_{n+1}|_{\infty} = \frac{1}{p} |x_{n} - d_n|_{\infty} < \frac{1}{p} (|x_n|_{\infty} + p) \le \frac{1}{2}|x_n|_{\infty} + 1 \le \frac{1}{2} M + 1 \le M. \] 
So we have (b) as well. From (a) and (b) it is easy to conclude: $[-M, M]$ has finite intersection with $\frac{1}{b}\Z$, so the $x_n$ cannot all be distinct.
\end{sol}

\begin{sol}{prob:limitabsvalue}\index{limits (sequential)} Observe that 
\[ |x_n|- |x_n-x| \le |x_n-(x_n-x)| = |x| = |x_n+(x-x_n)| \le |x_n| + |x-x_n|. \]
Hence, 
\[ -|x-x_n| \le |x|-|x_n| \le |x-x_n|. \]
Since $|x_n-x|\to 0$, the squeeze theorem implies that $|x| -|x_n| \to 0$. Therefore, $|x_n|$ converges to $|x|$.

Suppose now that $|\cdot|$ is non-Archimedean and that $x_n\to x$, where $x\ne 0$. The limit definition guarantees that $|x_n-x| < |x|$ for all sufficiently large $n$. For these $n$, ``survival of the greatest'' yields $|x_n| = |(x_n-x)+x| = |x|$.
\end{sol}

\begin{sol}{prob:45} By Problem \ref{prob:34} we can choose a prime $p$ with $|p|< 1$. Then whenever $n$ is an integer not divisible by $p$, the integers $p$ and $n$ are relatively prime and  Problem \ref{prob:35} tells us that $|n|_p=1$. 

Write $|p|= p^{-c}$ with $c> 0$. Given $x \in \Q^{\times}$, we can express $x=p^{v_p(x)} a/b$ where $a$ and $b$ are integers not divisible by $p$. Then 
\[ |x| = |p|^{v_p(x)} |a| |b|^{-1} = |p|^{v_p(x)} = p^{-c v_p(x)} = (p^{-v_p(x)})^{c} = |x|_p^{c}. \]
Of course, we also have $|0| = 0 = |0|_p^c$. So $|\cdot| = |\cdot|_p^{c}$.
\end{sol}

\begin{sol}{prob:46} From Exercise \ref{prob:33}, $|r| > 1$ for each integer $r\ge 2$. So there are indeed real numbers $c,d >0$ with $|2| = 2^c$ and $|3| = 3^d$. 

If we write $2^n = \sum_{i=0}^{m} \epsilon_i 3^i$, with each $\epsilon_i \in \{0,1,2\}$ and $\epsilon_m > 0$, then 
\begin{align*} 2^{cn} = |2|^n = |2^n| &\le \sum_{i=0}^{m} |\epsilon_i| |3|^i \le \max\{|1|,|2|\} \frac{|3|^{m+1}-1}{|3|-1} \\&\le \frac{\max\{|1|,|2|\} |3|}{|3|-1} \cdot |3|^{m} = \frac{|2|\cdot |3|}{|3|-1} 3^{dm},
\end{align*}
proving the claimed inequality with $B := \frac{|2|\cdot |3|}{|3|-1}$. As $\epsilon_m > 0$, we have $2^n \ge 3^m$, and thus $3^{dm} \le 2^{dn}$. Therefore, $2^{cn} \le B \cdot 2^{dn}$, and $2^{(c-d)n} \le B$. Since $n$ may be taken arbitrarily large, it follows that $c\le d$.

We could have run the argument with the roles of $2$ and $3$ reversed. Writing $3^n$ in base $2$ and reasoning analogously would lead to the inequality $3^{dn} \le B' 3^{cn}$, where $B' = \frac{|2|}{|2|-1}$. We would then conclude that $d \le c$. Thus, $c=d$.
\end{sol}

\begin{sol}{prob:47} The arguments of Problem \ref{prob:46} apply equally well with $3$ replaced by an arbitrary integer $r \ge 3$. Writing $|2| = 2^{c}$ and $|r| = r^{d}$, we find that for each positive integer $n$,
\[ 2^{cn} \le B\cdot 2^{dn}\quad\text{while}\quad r^{dn} \le B'\cdot r^{cn} \]
for the constants $B = \frac{\max\{ |2|,\dots,|r-1|\} |r|}{|r|-1}$ and $B' = \frac{|2|}{|2|-1}$. These two inequalities imply $c\le d$ and $d\le c$ (respectively), yielding $c=d$. 

It follows that $|r| = r^{c}$ for every integer $r> 1$. That equality holds trivially when $r=1$, and so $|\cdot| = |\cdot|^{c}$ on all of $\Z^{+}$. Writing each $x \in \Q^{\times}$ in the form $x= \pm \frac{a}{b}$ where $a,b \in \Z^{+}$, we deduce that
\[ |x| = |a| \cdot |b|^{-1} = a^{c} b^{-c} = (a b^{-1})^{c} = |x|_{\infty}^{c}.\]
Of course $|0| = 0 = |0|_{\infty}^{c}$ as well, and so $|\cdot| = |\cdot|_{\infty}^c$.
\end{sol}

\begin{challenge} For which positive real numbers $c$ is $|\cdot|_{\infty}^{c}$ an absolute value on $\Q$? Now let $p$ be prime. For which positive real numbers $c$ is $|\cdot|_p^{c}$ an absolute value on $\Q$?

% Let $c$ be a positive real number. \mbox{ }
% \vspace{-0.12in} 
% \begin{enumerate}
% \item[(a)] Show that if $p$ is prime, then $|\cdot|_{p}^{c}$ is an absolute value on $\Q$.
% \item[(b)] Prove that $|\cdot|_{\infty}^{c}$ is an absolute value on $\Q$ $\Longleftrightarrow$ $c\le 1$.
% \end{enumerate}
\end{challenge}

\begin{sol}{prob:48}\index{equivalent absolute values}\index{absolute value!equivalence} We begin with a simple but useful observation: If $(K,|\cdot|)$ is any valued field, and $x \in K$, then
\[ x^n \to 0 \text{ in $K$}\Longleftrightarrow |x^n-0| \to 0\text{ in $\R$} \Longleftrightarrow |x|< 1.\]

Now suppose for a contradiction that $|\cdot|$ and $|\cdot|'$ are equivalent absolute values on $K$ with $|\cdot|$ Archimedean and $|\cdot|'$ non-Archimedean. By Problem \ref{prob:34}, $|2| > 1$, and so $|\frac{1}{2}| < 1$. Hence, $(\frac{1}{2})^n \to 0$ in $(K,|\cdot|)$. Since $|\cdot|$ and $|\cdot|'$ are equivalent, $(\frac{1}{2})^n \to 0$ in $(K,|\cdot|')$ as well, so that $|\frac{1}{2}|' < 1$. But then $|2|'> 1$, contradicting that $|2|' = |1+1|' \le \max\{|1|'
,|1|'\} = 1$.  
\end{sol}

\begin{sol}{prob:49}\index{absolute value!classification of all abs.~values on $\Q$ (Ostrowski's theorem)}\index{Ostrowski's theorem}  Let $|\cdot|$ be a nontrivial absolute value on $\Q$. If $|\cdot|$ is Archimedean, then $|\cdot| = |\cdot|_{\infty}^{c}$ for some $c>0$ (Problem \ref{prob:47}). In this case, $|\cdot|$ is equivalent to $|\cdot|_{\infty}$, since
\begin{align*} x_n\to x\text{ in $(\Q,|\cdot|)$} &\Longleftrightarrow |x_n-x| \to 0 \\ &\Longleftrightarrow |x_n-x|_{\infty} \to 0 \Longleftrightarrow x_n \to x \text{ in $(\Q,|\cdot|_{\infty})$}.\end{align*}
If $|\cdot|$ is non-Archimedean, then $|\cdot|=|\cdot|_p^{c}$ for some prime $p$ and some $c>0$. In this case $|\cdot|$ is equivalent to $|\cdot|_p$ (same reasoning as displayed above).

It remains to show that none of $|\cdot|_{\infty}, |\cdot|_2, |\cdot|_3, |\cdot|_5\dots$ are equivalent. From Exercise \ref{prob:48}, $|\cdot|_{\infty}$ is not equivalent to any of the others, since $|\cdot|_{\infty}$ is Archimedean while $|\cdot|_p$ is non-Archimedean. If $p$ and $q$ are distinct primes, then $p^n \to 0$ in $(\Q,|\cdot|_p)$ but $p^n \not\to 0$ in $(\Q,|\cdot|_q)$. So $|\cdot|_p$ and $|\cdot|_q$ are inequivalent.
\end{sol}

\begin{challenge}[classifying absolute values on $\F_p(T)$]\index{absolute value!classification of all abs.~values on $\F_p(T)$} Fix a prime $p$, and let $\bmPi$ denote the collection of all monic irreducible polynomials $\pi \in \F_p[T]$. Let $|\cdot|$ be an absolute value on $\F_p(T)$.
\vspace{-0.12in}
\begin{enumerate}
\item[(a)] Suppose that $|T| > 1$. Prove that $|F| = |T|^{\deg{a}-\deg{b}}$ for all $F =\frac{a}{b}\in \F_p(T)^{\times}$.\item[(b)] Now assume that $|T| \le 1$. Prove that $|F|\le 1$ for all $F \in \F_p[T]$. Assuming $|\cdot|$ is nontrivial, show that there is a unique $\pi \in \raisebox{-0.25pt}{$\bmPi$}$ with $|\pi| < 1$. Furthermore, $|F| = |\pi|^{v_\pi(F)}$ for all $F \in \F_p(T)^{\times}$. (Recall that when $F=\pi^{v} \frac{a}{b}$ for $a,b \in \F_p[T]$ coprime to $\pi$, we are setting $v_{\pi}(F):=v$.)
\item[(c)] State and prove an $\F_p(T)$-analogue of the assertion in Problem \ref{prob:49}. 
\end{enumerate}

\end{challenge}

\begin{sol}{prob:CC} NO. Suppose $x_0, \dots, x_{p}$ are $p+1$ (distinct) equidistant rational numbers. Translating each by $-x_0$, we can assume $x_0= 0$. Next, scaling the $x_i$ by the same power of $p$, we can assume that each $x_i \in \Z_{(p)}$ and that at least one of them, say $x_1$, is not in $p\Z_{(p)}$. Then $|x_1-x_0|_p = |x_1|_p= 1$. 

Each of $x_1,\dots,x_{p}$ is congruent modulo $p\Z_{(p)}$ to one of $0,1,2,3,\dots,p-1$ (see the solution to Problem \ref{prob:44}). Since the $x_i$ are equidistant from one another, 
\[ |x_j|_p  =  |x_j-x_0|_p = |x_1-x_0|_p=1 \] for all $j=1,2,\dots,p$. Therefore, each of $x_1,\dots,x_p$ is congruent to one of $1,2,\dots,p-1 \pmod {p\Z_{(p)}}$. But then two of $x_1,\dots,x_{p}$, say $x_i$ and $x_j$, coincide modulo $p\Z_{(p)}$ (Pigeonhole Principle). For these two, 
\[ |x_i-x_j|_p \le \frac{1}{p} < 1 = |x_1-x_0|_p. \]
Contradiction!
\end{sol}


\begin{sol}{prob:50}\index{harmonic number} Notice that $H_n - \frac{1}{p}H_{m} = \sum_{1 \le k \le n} \frac{1}{k} - \sum_{1 \le r \le m}\frac{1}{pr} = \sum_{1 \le k \le n,~p\nmid k}\frac{1}{k} \in \Z_{(p)}$. Hence, $|H_n - \frac{1}{p}H_m|_p \le 1 \le |H_m|_p < p|H_m|_p = |\frac{1}{p}H_m|_p$. Now ``survival of the greatest'' gives us
\[ |H_n|_p = \left|\frac{1}{p}H_m + \left(H_n - \frac{1}{p}H_m\right)\right|_p = \left|\frac{1}{p}H_m\right|_p = p |H_m|_{p} \]
\end{sol}

\begin{sol}{prob:51} Let $p$ be a prime. Suppose we happen to have in hand a nonnegative integer $k$ with $|H_m|_{p} \ge 1$ for all $m$ in the range $p^k \le m < p^{k+1}$. Problem \ref{prob:50} allows us to conclude that $|H_n|_{p} \ge p$ whenever $p^{k+1} \le n < p^{k+2}$. Repeating the reasoning, $|H_n|_{p} \ge p^2$ whenever $p^{k+2} \le n < p^{k+3}$. In general, $|H_n|_p \ge p^{j}$ for all $n\ge p^{k+j}$ (for $j=0,1,2,3,\dots$). Therefore, $|H_n|_{p}\to\infty$. 

One can check with a simple computer program that $k=2$ satisfies our hypothesis both when $p=3$ and when $p=5$. So $|H_n|_{3}\to\infty$ and $|H_n|_{5}\to\infty$. 

\begin{rmk} For an in-depth examination of $|H_n|_{p}$, see \cite{boyd}. 

Several open questions persist regarding the numerators and denominators of the harmonic numbers. Here is one that appears deceptively simple: Are there infinitely many $n$ for which the denominator of $H_n$ (in lowest terms) is the least common multiple of $1,2,3,\dots, n$?
\end{rmk}

\end{sol}

\let\oldaddcontentsline\addcontentsline
\renewcommand{\addcontentsline}[3]{}
\begin{thebibliography}{11}
        \bibitem{boyd}  D.\,W. Boyd,
\emph{A $p$-adic study of the partial sums of the harmonic series}. Experiment. Math. \textbf{3} (1994), 287--302.

\bibitem{IR} K. Ireland and M. Rosen, \emph{A classical introduction to modern number theory}, second ed., Graduate Texts in Mathematics, vol. 84, Springer-Verlag, New York, 1990.
    \end{thebibliography}
\let\addcontentsline\oldaddcontentsline

