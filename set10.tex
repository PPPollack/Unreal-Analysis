\setcounter{chapter}{11}
\chapter*{$p$-Set \#11}
\addcontentsline{toc}{chapter}{Set \#11}
\markboth{Set \#11}{Set \#11}
\label{chap:skolem}

\vspace{-0.4in}
%\setlength{\epigraphwidth}{0.51\textwidth}
% \epigraph{Ordinary analysis has amassed a great stock of identities between power series. Many of these are valid in $p$-adic analysis too. But here an identity between power series yields congruences between partial sums.}{Max Zorn}
%\setlength{\epigraphwidth}{0.49\textwidth}
\epigraph{In 1926, when he was nearly 40 years old, Skolem obtained his
doctorate\dots The somewhat advanced age
has the following explanation. In their younger years, Viggo Brun and
Skolem agreed that neither of them would bother to obtain the degree of
Doctor, probably feeling that, in Norway, it served no useful function
in the education of a young scientist. But in the middle twenties a
younger generation of Norwegian mathematician emerged. It seems
that Skolem then felt he too ought to fulfil the formal requirement
of having a doctorate, and he ``obtained permission'' from Brun to
submit a thesis.}{Jens Erik Fenstad}

\vspace{0.12in}
\section*{I \dots Have \dots the \dots Power\dots (Series)}
\begin{prob}[$p$-adic lumber theory]\label{prob:padiclumber1}\index{p-adic@$p$-adic logarithm!definition and convergence on $1+p\Z_p$}\index{logarithm, on $\Q_p$|see{$p$-adic logarithm}} Recall from calculus that \[ \log{x} = \log(1+(x-1)) = \sum_{k\ge 1} \frac{(-1)^{k-1}}{k} (x-1)^k \quad\text{whenever} \quad |x-1| < 1.\] This motivates us to define, for each prime $p$,
\[ \log_p(T) = \sum_{k\ge 1} \frac{(-1)^{k-1}}{k} (T-1)^{k} \in \Q_p[[T-1]]. \] Show: $\log_p{x}$ converges for all $x \in \Q_p$ with $|x-1|_p < 1$.
\end{prob}

\testrule
Passing the familiar identity $\log{xy} = \log{x} +\log{y}$ (valid for $x,y \in \R^{+}$) from $\R$ to $\Q_p[[X-1,Y-1]]$, and then on to $\Q_p$, one can show that
\begin{equation}\tag{$\dagger$} \log_p{x} + \log_p{y} = \log_p(xy) \quad\text{whenever $x,y \in 1 +p\Z_p$}.\end{equation}
Filling in the details here is a bit finicky. If you enjoy this kind of work (you know who you are\dots), give it a try!
\testruletwo

\begin{prob}\label{prob:padiclumber2} Go ahead and assume (\textdagger).

Show: $\displaystyle\sum_{k=1}^{\infty}\left(1+\frac{1}{2^k} + \frac{1}{3^k} + \dots + \frac{1}{(p-1)^k}\right) \frac{p^k}{k}=0$ in $\Q_p$. \\
In particular ($p=2$):\quad $\displaystyle\sum_{k=1}^{\infty}\frac{2^k}{k}=0$ in $\Q_2$.\footnote{``Ordinary analysis has amassed a great stock of identities between power series. Many of these are valid in $p$-adic analysis too. But here an identity between power series yields congruences between partial sums.'' --- Max Zorn}



% {\scriptsize Put $\Exp_p(T)=\sum_{k\ge 0} \frac{T^k}{k!}$. Suppose $p$ is odd (the situation is slightly more complicated when $p=2$). Then $\Exp_p(T)$ converges for $x\in p\Z_p$, and for these same $x$ one has $\Exp_p(\log_p(1+x)) = 1+x$ and $\log_p(\Exp_p(x)) = x$. Also, $\Exp_p(x+y)=\Exp_p(x)\cdot \Exp_p(y)$ for all $x,y\in p\Z_p$.}
\end{prob}



\begin{prob}[derive responsibly!]\label{prob:fakeproofkoblitz} Critique the following proof that $\pi$ is irrational: Suppose that $\pi = \frac{a}{b}$, where $a$ and $b$ are positive integers. Let $p$ be an odd prime not dividing $a$. Then $$0 = \sin(pb \pi) = \sin(ap) = \sum_{k\ge 0} (-1)^k \frac{(ap)^{2k+1}}{(2k+1)!}, $$ 
where this last series converges in $\Q_p$. Therefore,
\[ p^{-1} = |-ap|_p = \left|\sum_{k\ge 0} (-1)^k \frac{(ap)^{2k+1}}{(2k+1)!} -ap\right|_p = \left|\sum_{k\ge 1} (-1)^k \frac{(ap)^{2k+1}}{(2k+1)!}\right|_p\le p^{-2}. \]
Contradiction! \label{ex:pi}
\end{prob}




% \begin{prob} In courses on rigorous calculus, it is shown that when $x$ is  real  with $|x| < 1$, the series $\sum_{k\ge 0} \frac{\frac{1}{2}(\frac{1}{2}-1)\cdots (\frac{1}{2}-(k-1))}{k!} x^k$ converges to the positive square root of  $1+x$. Show that for $p$ odd and $x \in p\Z_p$, that same series converges to the square root of $1+x$ belonging to $1+p\Z_p$. Hence, $\sum_{k\ge 0} \frac{\frac{1}{2}(\frac{1}{2}-1)\cdots (\frac{1}{2}-(k-1))}{k!} (9/16)^{k}$ converges to $5/4$ in $(\R,|\cdot|)$ but to $-5/4$ in $(\Q_3,|\cdot|_3)$.
% \end{prob}

\section*{Strassmann Series}

\begin{prob}\label{prob:uniquezero} Let $F(T) = 1 + T + (pT)^2 + (pT)^4 + (pT)^8 + (pT)^{16} + \dots$. There is exactly one $x\in \Z_p$ with $F(x)=0$.
\end{prob}

\vspace{-0.22in}
\testrule
It will be convenient for the next exercise, and certain others afterward, to name the coefficients of the \textsf{falling factorial} $T(T-1) \cdots (T-(N-1))$. For nonnegative integers $N$ and $K$, we let $s(N,K)$ be the integer defined by the formal equality \[ T(T-1)\cdots(T-(N-1)) = \sum_{K=0}^{\infty} s(N,K)  T^K.\]
For example, when $N=5$, we have $T(T-1)(T-2)(T-3)(T-4) = 24T - 50T^2 + 35T^3 - 10T^4 + T^5$, so that
\begin{multline*} s(5,0)=0, \quad s(5,1) = 24, \quad s(5,2) = -50, \quad s(5,3) = 35, \\ s(5,4)=-10, \quad s(5,5) = 1, \quad\text{and}\quad s(5,K) = 0 \text{ for $K > 5$}.\end{multline*}
When $N=0$, the product $T(T-1)\cdots (T-(N-1))$ is empty and assigned  the value $1$; hence, $s(0,0)=1$ and $s(0,K)=0$ for $K>0$. The $s(N,K)$ are known as \textsf{Stirling numbers of the first kind}. \index{$s(N,K)$, Stirling numbers of the first kind} Don't let the fancy name fool you; while the Stirling numbers are important in combinatorics, for us they play only a notational role.
\testruletwo

\begin{prob}[$p$-adically interpolating $(1+a)^{x}$]\label{ex:binomialseries}\label{prob:binomialseries}\index{Strassmann series!representing $(1+a)^n$}
Let $p$ be an odd prime, and let $a \in p\Z_p$. If $n$ is a nonnegative integer, then 
\begin{align*} (1+a)^{n} &= \sum_{k=0}^{\infty} n(n-1)(n-2)\cdots(n-(k-1)) \frac{a^k}{k!} \\
&= \sum_{k=0}^{\infty} \left(\sum_{j=0}^{k} s(k,j)n^j\right)\frac{a^k}{k!} = \sum_{j=0}^{\infty} C_{a,j} n^j,\end{align*}
where
\[ C_{a,j} := \sum_{k\ge j} s(k,j) \frac{a^k}{k!}. \]
Moreover, $|C_{a,j}|_p \to 0$ as $j\to\infty$. 

{\scriptsize Your job: Fill in the missing details. In particular, justify the swapping of the sums on $k$ and $j$.}

\end{prob}

%\testrule
\underline{\textsc{Notation}}. For future use, we let $$\Binom(1+a;T) := \sum_{j=0}^{\infty} C_{a,j} T^j \in \Q_p[[T]].\index{$\Binom(1+a;T)$|see{Strassmann series representing $(1+a)^n$}}$$ As you have just shown, $\Binom(1+a;T)$ is a Strassmann series satisfying $\Binom(1+a;n) = (1+a)^n$ for all nonnegative integers $n$.
%\testrule




%\vspace{-0.25in}

%See Gouvea's text for a precise statement.
\vspace{-0.15in}
\section*{Zeros of Linear Recurrence Sequences}

Let $\{x_n\}_{n\ge 0}$ be a sequence of integers satisfying a linear recurrence
\[ x_{n} = a_1 x_{n-1} + a_2 x_{n-2} + \dots + a_d x_{n-d} \quad\text{ for $n=d,d+1,d+2,\dots$}, \]
where $a_1, \dots, a_d\in \Z$, and $a_d \ne 0$. Put 
\[ A = \left[\begin{matrix} 
0 & 1 & 0  & \hdots & 0\\
0 & 0 & 1  & \hdots & 0 \\
\vdots & \vdots & \vdots & \ddots & \vdots \\
0 & 0 & 0 & \hdots & 1 \\
a_d & a_{d-1} & a_{d-2} & \hdots & a_1
\end{matrix}\right]\qquad\text{and}\qquad \textbf{v} = \left[ 
\begin{matrix}
x_0 \\
x_1 \\
x_2 \\
\vdots \\
x_{d-1}
\end{matrix}
\right], \quad \textbf{e} = 
\left[\,\begin{matrix}
1 \\
0 \\
0 \\
\vdots \\
0
\end{matrix}\,\right].
\]
\begin{prob}\label{prob:matrixmult} $x_{n} = \langle A^n\vv, \ee\rangle$.\end{prob}

\begin{prob}\label{prob:pinvertible} If $p\nmid a_d$, then $A$ is invertible over $\F_p$.
\end{prob}

\begin{prob}\label{prob:matrixexpression} Let $p$ be as in Problem \ref{prob:pinvertible}, and let $k$ be the order of $A$ in $\GL(d,\F_p)$, so that $A^k = \mathrm{Id} +pB$ for some integer matrix $B$. If $n=km+r$, where $r \in \{0,1,\dots,k-1\}$, then 
\[ x_n = \sum_{0 \le j \le m} \binom{m}{j} p^j \langle A^r B^j\vv,\ee\rangle.  \]
\end{prob}

\begin{prob}[$p$-adically interpolating $x_{km+r}$]\label{prob:matrixbinomial} Continue with the above notation and assumptions but assume additionally that $p$ is odd. For each fixed $r\in\{0,1,\dots,k-1\}$, there is a Strassmann series $F_{k,r}(T)$ with $F_{k,r}(m) = x_{km+r}$ for every integer $m\ge 0$.\index{Strassmann series!representing terms of a linear recurrence}
\end{prob}

\begin{prob}\label{prob:skdichotomy} For each fixed $r\in\{0,1,\dots,k-1\}$, either $x_{n}=0$ for all nonnegative integers $n\equiv r\pmod{k}$, or $x_{n}=0$ for only finitely many $n\equiv r\pmod{k}$.  

Hence: The set of $n$ with $x_n=0$ is the union of a finite set and a finite collection of residue classes. \textbf{(Skolem's Theorem)}\index{Skolem--Mahler--Lech theorem}
\end{prob}

\begin{prob}\label{prob:easyrecurrence} Give an example of an integer linear recurrence sequence, not identically $0$, with the property that $x_n=0$ for infinitely many $n$.
\end{prob}

\begin{prob}[Mahler]\label{prob:mahlerSML} Does Skolem's Theorem hold for recurrence sequences over $\Q$? (``Over $\Q$'' means that $a_1,\dots,a_d$ and $x_0,\dots,x_{d-1}$ belong to $\Q$.) Over an arbitrary finite extension of $\Q$ (number field)?
\end{prob}


